\documentclass{article}
\usepackage[top=3cm, bottom=3cm, outer=3cm, inner=3cm]{geometry}
\usepackage{multicol}
\usepackage{graphicx} % Required for inserting images
\usepackage{url}
%\usepackage{cite}
\usepackage{hyperref}
\usepackage{array}
\usepackage{fontenc}

%\usepackage{multicol}
\newcolumntype{x}[1]{>{\centering\arraybackslash\hspace{0pt}}p{#1}}
\usepackage{natbib}
\usepackage{pdfpages}
\usepackage{multirow}
\usepackage[normalem]{ulem}
\useunder{\uline}{\ul}{}
\usepackage{svg}
\usepackage{xcolor}
\usepackage{listings}
\lstdefinestyle{ascii-tree}{
    literate={├}{|}1 {─}{--}1 {└}{+}1 
  }
\lstset{basicstyle=\ttfamily,
  showstringspaces=false,
  commentstyle=\color{red},
  keywordstyle=\color{blue}
}
%\usepackage{booktabs}
\usepackage{caption}
\usepackage{subcaption}
\usepackage{float}
\usepackage{array}

\newcolumntype{M}[1]{>{\centering\arraybackslash}m{#1}}
\newcolumntype{N}{@{}m{0pt}@{}}

%%%%%%%%%%%%%%%%%%%%%%%%%%%%%%%%%%%%%%%%%%%%%%%%%%%%%%%%%%%%%%%%%%%%%%%%%%%%
%%%%%%%%%%%%%%%%%%%%%%%%%%%%%%%%%%%%%%%%%%%%%%%%%%%%%%%%%%%%%%%%%%%%%%%%%%%%
\newcommand{\itemEmail}{aticonam@ulasalle.edu.pe}
\newcommand{\itemStudent}{Andrea Ticona Mamani}
\newcommand{\itemCourse}{Compiladores}
\newcommand{\itemCourseCode}{3.5.6.21}
\newcommand{\itemSemester}{V}
\newcommand{\itemUniversity}{Universidad La Salle}
\newcommand{\itemFaculty}{Facultad de Ingenierías}
\newcommand{\itemDepartment}{Departamento de Ingeniería y Matemáticas}
\newcommand{\itemSchool}{Carrera Profesional de Ingeniería de Software}
\newcommand{\itemAcademic}{2024 - B}
\newcommand{\itemOutput}{09 Septiembre 2024}
%\newcommand{\itemPracticeNumber}{05}
\newcommand{\itemTheme}{Implementación del analizador léxico}
%%%%%%%%%%%%%%%%%%%%%%%%%%%%%%%%%%%%%%%%%%%%%%%%%%%%%%%%%%%%%%%%%%%%%%%%%%%%
%%%%%%%%%%%%%%%%%%%%%%%%%%%%%%%%%%%%%%%%%%%%%%%%%%%%%%%%%%%%%%%%%%%%%%%%%%%%
\usepackage[english,spanish]{babel}
\usepackage[utf8]{inputenc}
\AtBeginDocument{\selectlanguage{spanish}}
\renewcommand{\figurename}{Figura}
\renewcommand{\refname}{Referencias}
\renewcommand{\tablename}{Tabla} %esto no funciona cuando se usa babel
\AtBeginDocument{%
	\renewcommand\tablename{Tabla}
}

\usepackage{fancyhdr}
\pagestyle{fancy}
\fancyhf{}
\setlength{\headheight}{30pt}
\renewcommand{\headrulewidth}{1pt}
\renewcommand{\footrulewidth}{1pt}
\fancyhead[L]{\raisebox{-0.2\height}{\includegraphics[width=3cm]{img/logo_salle.png}}}
\fancyhead[C]{\fontsize{7}{7}\selectfont	\itemUniversity \\ \itemFaculty \\ \itemDepartment \\ \itemSchool \\ \textbf{\itemCourse}}
\fancyfoot[L]{Andrea Janeth Ticona Mamani}
\fancyfoot[C]{Página \thepage}
\fancyfoot[R]{\itemCourse}

% para el codigo fuente
\usepackage{listings}
\usepackage{color, colortbl}
\definecolor{dkgreen}{rgb}{0,0.6,0}
\definecolor{gray}{rgb}{0.5,0.5,0.5}
\definecolor{mauve}{rgb}{0.58,0,0.82}
\definecolor{codebackground}{rgb}{0.95, 0.95, 0.92}
\definecolor{tablebackground}{rgb}{0.0, 0.45, 0.63}

\lstset{frame=tb,
	language=bash,
	aboveskip=3mm,
	belowskip=3mm,
	showstringspaces=false,
	columns=flexible,
	basicstyle={\small\ttfamily},
	numbers=none,
	numberstyle=\tiny\color{gray},
	keywordstyle=\color{blue},
	commentstyle=\color{dkgreen},
	stringstyle=\color{mauve},
	breaklines=true,
	breakatwhitespace=true,
	tabsize=3,
	backgroundcolor= \color{codebackground},
}
\begin{document}

\vspace*{10px}
	
	\begin{center}	
		\fontsize{17}{17} \textbf{ $Practica$}
	\end{center}
	\centerline{\textbf{\Large Tema: \itemTheme}}
 
 \begin{table}[H]
		\begin{tabular}{|x{4.7cm}|x{4.8cm}|x{4.8cm}|}
			\hline 
			\rowcolor{tablebackground}
			\color{white} \textbf{Estudiante} & \color{white}\textbf{Escuela}  & \color{white}\textbf{Asignatura}   \\
			\hline 
			{\itemStudent \par \itemEmail} & \itemSchool & {\itemCourse \par Semestre: \itemSemester \par Código: \itemCourseCode}     \\
			\hline 			
		\end{tabular}
	\end{table}		

 \begin{table}[H]
		\begin{tabular}{|x{4.7cm}|x{4.8cm}|x{4.8cm}|}
			\rowcolor{tablebackground}
			%\color{white}\textbf{Examen Parcial} & 
                %\color{white}\textbf{Tema}  & 
                \color{white}\textbf{Duración} & \color{white}\textbf{Semestre académico} &  
                \color{white}\textbf{Fecha de entrega}   \\
			
			\hline 
			%\itemPracticeNumber & 
                02 horas & \itemAcademic & \itemOutput  \\
			\hline 
		\end{tabular}
	\end{table}
 
 \tableofcontents

 \section{Ejercicios}
    \begin{itemize}
        \item Implemente el analizador léxico para el lenguaje propuesto. Su programa deberá leer el código fuente de archivo en disco (debe proporcionar varios ejemplos) y luego deberá mostrar todos los tokens de manera similar al ejemplo mostrado en la Sección 5(del archivo que el docente nos envió). Asegurese, de que los tokens retornados esten en una lista de objetos o diccionario.
    \end{itemize}
\subsection{Palabras Reservadas}
\begin{itemize}
    \item En esta seccion se definen las palabras claves del lenguaje propuesto(Python en español con delimitacion con llaves)
\end{itemize}
\lstinputlisting[language=Python, caption={Palabras reservadas},numbers=left,]{src/reservadas.py}

\subsection{Tokens}
\begin{itemize}
    \item En esta seccion se definen los tokens del lenguaje propuesto
\end{itemize}
\lstinputlisting[language=Python, caption={Token},numbers=left,]{src/tokens.py}
\begin{itemize}
    \item En esta seccion se definen los tokens para los operadores del lenguaje propuesto
\end{itemize}
\lstinputlisting[language=Python, caption={Token para Operadores},numbers=left,]{src/tokensoperadores.py}
\begin{itemize}
    \item Aqui se actualiza la lista de tokens, combinando la lista original de tokens con los valores de palabras\_reservadas.
    \item Luego definimos que ignore los espacios en blanco y las tabulaciones cuando procesa el texto de entrada.
\end{itemize}
\lstinputlisting[language=Python, caption={Union y uso del ignore},numbers=left,]{src/otro.py}

\subsection{Expresiones Regulares}
\begin{itemize}
    \item En esta seccion se define las expresiones regulares que se usaran.
\end{itemize}
\subsubsection{Expresion regular para una variable}
\lstinputlisting[language=Python, caption={Variable},numbers=left,]{src/variable.py}
\subsubsection{Expresion regular para un int}
\lstinputlisting[language=Python, caption={Enteros},numbers=left,]{src/entero.py}
\subsubsection{Expresion regular para un float}
\lstinputlisting[language=Python, caption={Decimales},numbers=left,]{src/decimal.py}
\subsubsection{Expresion regular para un string}
\lstinputlisting[language=Python, caption={Cadena},numbers=left,]{src/cadena.py}
\subsubsection{Expresion regular para una nueva linea}
\begin{itemize}
    \item Cada vez que el analizador encuentre una nueva linea, la funcion actualiza el número de linea del lexer y lo incrementa en la cantidad de nuevas lineas.
\end{itemize}
\lstinputlisting[language=Python, caption={Nueva Linea},numbers=left,]{src/cadena.py}
\subsubsection{Expresion regular para un comentario lineal}
\begin{itemize}
    \item Se reconoce los comentarios de una sola linea.
\end{itemize}
\lstinputlisting[language=Python, caption={Comentario},numbers=left,]{src/comentariolineal.py}
\subsubsection{Expresion regular para un comentario en bloque}
\begin{itemize}
    \item Se reconoce los comentarios en bloque.
\end{itemize}
\lstinputlisting[language=Python, caption={Comentario en bloque},numbers=left,]{src/comentariobloque.py}
\subsubsection{Expresion regular para controlar un error lexico}
\begin{itemize}
    \item Cuando el analizador encuentre un caracter que no coincide con ninguno de los tokens definidos, se imprimira el mensaje de error y nos indicara cual fue el caracter no reconocido, sin embargo con t\.lexer\.skip(1) le dice a lexer que salte 1 caracter y continuar con el analisis.
\end{itemize}
\lstinputlisting[language=Python, caption={Error lexico},numbers=left,]{src/error.py}

\subsection{Lectura del archivo en disco}
\lstinputlisting[language=Python, caption={Funcion para leer el archivo en disco},numbers=left,]{src/lecturadearchivos.py}
\begin{itemize}
    \item El código lee un archivo desde el disco, procesa su contenido y mostrara todos los tokens que este encuentre. Se hace uso de un try para poder manejar correctamente la existencia del archivo.
    \item En la linea 5, se pasa el contenido del archivo al lexer para que lo analice.
    \item En la linea 7 se crea una lista llamada "diccionario" para almacenar los tokens.
    \item En la linea 13, cada token que encuentre el lexer se almacena en la lista diccionario. Cada token guarda la siguiente información: 
    \begin{itemize}
        \item Tipo de token
        \item Valor
        \item Número de línea
        \item Posición
    \end{itemize}
    \item En la linea 26, se inicializa el lexer.
     \item En la linea 28, se llama a la función analizar\_archivo y se llama a la carpeta donde estan los archivos(esta carpeta esta ubicado dentro de la carpeta del proyesto) y se coloca el nombre del archivo, como ejemplo el archivo que estaba ejecutando se llama "holamundo", siempre con terminacion txt.
\end{itemize}

\subsection{Ejemplos de funcionalidad}

\lstinputlisting[language=Python, caption={Hola mundo},numbers=left,]{src/holamundo.py}
\begin{lstlisting}[language=bash,caption={Tokens del Hola Mundo}][H]
{'Tipo de token ': 'CADENA', 'Valor ': 'cadena', 'Esta en la linea ': 2, 'En la posición ': 24}
{'Tipo de token ': 'VARIABLE', 'Valor ': 'x', 'Esta en la linea ': 2, 'En la posición ': 31}
{'Tipo de token ': 'OP_IGUAL', 'Valor ': '=', 'Esta en la linea ': 2, 'En la posición ': 33}
{'Tipo de token ': 'PALABRA', 'Valor ': 'Hola mundo', 'Esta en la linea ': 2, 'En la posición ': 35}
{'Tipo de token ': 'IMPRIMIR', 'Valor ': 'imprimir', 'Esta en la linea ': 3, 'En la posición ': 48}
{'Tipo de token ': 'PARIN', 'Valor ': '(', 'Esta en la linea ': 3, 'En la posición ': 57}
{'Tipo de token ': 'VARIABLE', 'Valor ': 'x', 'Esta en la linea ': 3, 'En la posición ': 58}
{'Tipo de token ': 'PARFIN', 'Valor ': ')', 'Esta en la linea ': 3, 'En la posición ': 59}
\end{lstlisting}

\lstinputlisting[language=Python, caption={Factorial},numbers=left,]{src/factorial.py}
\begin{lstlisting}[language=bash,caption={Tokens de Factorial}][H]
{'Tipo de token ': 'FUNCION', 'Valor ': 'funcion', 'Esta en la linea ': 2, 'En la posición ': 18}
{'Tipo de token ': 'VARIABLE', 'Valor ': 'factorial', 'Esta en la linea ': 2, 'En la posición ': 26}
{'Tipo de token ': 'PARIN', 'Valor ': '(', 'Esta en la linea ': 2, 'En la posición ': 35}
{'Tipo de token ': 'VARIABLE', 'Valor ': 'x', 'Esta en la linea ': 2, 'En la posición ': 36}
{'Tipo de token ': 'PARFIN', 'Valor ': ')', 'Esta en la linea ': 2, 'En la posición ': 37}
{'Tipo de token ': 'LLAIN', 'Valor ': '{', 'Esta en la linea ': 2, 'En la posición ': 38}
{'Tipo de token ': 'SI', 'Valor ': 'si', 'Esta en la linea ': 3, 'En la posición ': 44}
{'Tipo de token ': 'VARIABLE', 'Valor ': 'x', 'Esta en la linea ': 3, 'En la posición ': 47}
{'Tipo de token ': 'OP_IGUAL', 'Valor ': '=', 'Esta en la linea ': 3, 'En la posición ': 48}
{'Tipo de token ': 'NUMENTERO', 'Valor ': 1, 'Esta en la linea ': 3, 'En la posición ': 49}
{'Tipo de token ': 'O', 'Valor ': 'o', 'Esta en la linea ': 3, 'En la posición ': 51}
{'Tipo de token ': 'VARIABLE', 'Valor ': 'x', 'Esta en la linea ': 3, 'En la posición ': 53}
{'Tipo de token ': 'OP_IGUAL', 'Valor ': '=', 'Esta en la linea ': 3, 'En la posición ': 54}
{'Tipo de token ': 'NUMENTERO', 'Valor ': 0, 'Esta en la linea ': 3, 'En la posición ': 55}
{'Tipo de token ': 'LLAIN', 'Valor ': '{', 'Esta en la linea ': 3, 'En la posición ': 56}
{'Tipo de token ': 'RETORNAR', 'Valor ': 'retorna', 'Esta en la linea ': 4, 'En la posición ': 66}
{'Tipo de token ': 'NUMENTERO', 'Valor ': 1, 'Esta en la linea ': 4, 'En la posición ': 74}
{'Tipo de token ': 'LLAFIN', 'Valor ': '}', 'Esta en la linea ': 5, 'En la posición ': 80}
{'Tipo de token ': 'SINO', 'Valor ': 'sino', 'Esta en la linea ': 6, 'En la posición ': 86}
{'Tipo de token ': 'LLAIN', 'Valor ': '{', 'Esta en la linea ': 6, 'En la posición ': 90}
{'Tipo de token ': 'RETORNAR', 'Valor ': 'retorna', 'Esta en la linea ': 7, 'En la posición ': 100}
{'Tipo de token ': 'PARIN', 'Valor ': '(', 'Esta en la linea ': 7, 'En la posición ': 108}
{'Tipo de token ': 'VARIABLE', 'Valor ': 'x', 'Esta en la linea ': 7, 'En la posición ': 109}
{'Tipo de token ': 'OP_MULTI', 'Valor ': '*', 'Esta en la linea ': 7, 'En la posición ': 110}
{'Tipo de token ': 'VARIABLE', 'Valor ': 'factorial', 'Esta en la linea ': 7, 'En la posición ': 111}
{'Tipo de token ': 'PARIN', 'Valor ': '(', 'Esta en la linea ': 7, 'En la posición ': 120}
{'Tipo de token ': 'VARIABLE', 'Valor ': 'x', 'Esta en la linea ': 7, 'En la posición ': 121}
{'Tipo de token ': 'OP_MENOS', 'Valor ': '-', 'Esta en la linea ': 7, 'En la posición ': 122}
{'Tipo de token ': 'NUMENTERO', 'Valor ': 1, 'Esta en la linea ': 7, 'En la posición ': 123}
{'Tipo de token ': 'PARFIN', 'Valor ': ')', 'Esta en la linea ': 7, 'En la posición ': 124}
{'Tipo de token ': 'PARFIN', 'Valor ': ')', 'Esta en la linea ': 7, 'En la posición ': 125}
{'Tipo de token ': 'LLAFIN', 'Valor ': '}', 'Esta en la linea ': 8, 'En la posición ': 131}
{'Tipo de token ': 'VARIABLE', 'Valor ': 'numero', 'Esta en la linea ': 9, 'En la posición ': 133}
{'Tipo de token ': 'OP_IGUAL', 'Valor ': '=', 'Esta en la linea ': 9, 'En la posición ': 140}
{'Tipo de token ': 'NUMENTERO', 'Valor ': 7, 'Esta en la linea ': 9, 'En la posición ': 142}
{'Tipo de token ': 'VARIABLE', 'Valor ': 'resultado', 'Esta en la linea ': 10, 'En la posición ': 144}
{'Tipo de token ': 'OP_IGUAL', 'Valor ': '=', 'Esta en la linea ': 10, 'En la posición ': 154}
{'Tipo de token ': 'VARIABLE', 'Valor ': 'factorial', 'Esta en la linea ': 10, 'En la posición ': 156}
{'Tipo de token ': 'PARIN', 'Valor ': '(', 'Esta en la linea ': 10, 'En la posición ': 165}
{'Tipo de token ': 'VARIABLE', 'Valor ': 'numero', 'Esta en la linea ': 10, 'En la posición ': 166}
{'Tipo de token ': 'PARFIN', 'Valor ': ')', 'Esta en la linea ': 10, 'En la posición ': 172}
{'Tipo de token ': 'IMPRIMIR', 'Valor ': 'imprimir', 'Esta en la linea ': 11, 'En la posición ': 174}
{'Tipo de token ': 'PARIN', 'Valor ': '(', 'Esta en la linea ': 11, 'En la posición ': 182}
{'Tipo de token ': 'VARIABLE', 'Valor ': 'resultado', 'Esta en la linea ': 11, 'En la posición ': 183}
{'Tipo de token ': 'PARFIN', 'Valor ': ')', 'Esta en la linea ': 11, 'En la posición ': 192}
{'Tipo de token ': 'LLAFIN', 'Valor ': '}', 'Esta en la linea ': 12, 'En la posición ': 194}
\end{lstlisting}

\lstinputlisting[language=Python, caption={Fibonacci},numbers=left,]{src/fibonacci.py}
\begin{lstlisting}[language=bash,caption={Tokens de Fibonacci}][H]
{'Tipo de token ': 'FUNCION', 'Valor ': 'funcion', 'Esta en la linea ': 2, 'En la posición ': 18}
{'Tipo de token ': 'VARIABLE', 'Valor ': 'fibonacci', 'Esta en la linea ': 2, 'En la posición ': 26}
{'Tipo de token ': 'PARIN', 'Valor ': '(', 'Esta en la linea ': 2, 'En la posición ': 35}
{'Tipo de token ': 'VARIABLE', 'Valor ': 'x', 'Esta en la linea ': 2, 'En la posición ': 36}
{'Tipo de token ': 'PARFIN', 'Valor ': ')', 'Esta en la linea ': 2, 'En la posición ': 37}
{'Tipo de token ': 'LLAIN', 'Valor ': '{', 'Esta en la linea ': 2, 'En la posición ': 38}
{'Tipo de token ': 'SI', 'Valor ': 'si', 'Esta en la linea ': 3, 'En la posición ': 44}
{'Tipo de token ': 'VARIABLE', 'Valor ': 'x', 'Esta en la linea ': 3, 'En la posición ': 47}
{'Tipo de token ': 'OP_IGUAL', 'Valor ': '=', 'Esta en la linea ': 3, 'En la posición ': 49}
{'Tipo de token ': 'NUMENTERO', 'Valor ': 0, 'Esta en la linea ': 3, 'En la posición ': 51}
{'Tipo de token ': 'LLAIN', 'Valor ': '{', 'Esta en la linea ': 3, 'En la posición ': 52}
{'Tipo de token ': 'RETORNAR', 'Valor ': 'retorna', 'Esta en la linea ': 4, 'En la posición ': 62}
{'Tipo de token ': 'NUMENTERO', 'Valor ': 0, 'Esta en la linea ': 4, 'En la posición ': 70}
{'Tipo de token ': 'LLAFIN', 'Valor ': '}', 'Esta en la linea ': 5, 'En la posición ': 76}
{'Tipo de token ': 'SI', 'Valor ': 'si', 'Esta en la linea ': 6, 'En la posición ': 82}
{'Tipo de token ': 'VARIABLE', 'Valor ': 'x', 'Esta en la linea ': 6, 'En la posición ': 85}
{'Tipo de token ': 'OP_IGUAL', 'Valor ': '=', 'Esta en la linea ': 6, 'En la posición ': 87}
{'Tipo de token ': 'NUMENTERO', 'Valor ': 1, 'Esta en la linea ': 6, 'En la posición ': 89}
{'Tipo de token ': 'LLAIN', 'Valor ': '{', 'Esta en la linea ': 6, 'En la posición ': 90}
{'Tipo de token ': 'RETORNAR', 'Valor ': 'retorna', 'Esta en la linea ': 7, 'En la posición ': 100}
{'Tipo de token ': 'NUMENTERO', 'Valor ': 1, 'Esta en la linea ': 7, 'En la posición ': 108}
{'Tipo de token ': 'LLAFIN', 'Valor ': '}', 'Esta en la linea ': 8, 'En la posición ': 114}
{'Tipo de token ': 'SINO', 'Valor ': 'sino', 'Esta en la linea ': 9, 'En la posición ': 120}
{'Tipo de token ': 'LLAIN', 'Valor ': '{', 'Esta en la linea ': 9, 'En la posición ': 125}
{'Tipo de token ': 'VARIABLE', 'Valor ': 'fibo', 'Esta en la linea ': 10, 'En la posición ': 135}
{'Tipo de token ': 'OP_IGUAL', 'Valor ': '=', 'Esta en la linea ': 10, 'En la posición ': 140}
{'Tipo de token ': 'VARIABLE', 'Valor ': 'fibonacci', 'Esta en la linea ': 10, 'En la posición ': 142}
{'Tipo de token ': 'PARIN', 'Valor ': '(', 'Esta en la linea ': 10, 'En la posición ': 151}
{'Tipo de token ': 'VARIABLE', 'Valor ': 'x', 'Esta en la linea ': 10, 'En la posición ': 152}
{'Tipo de token ': 'OP_MENOS', 'Valor ': '-', 'Esta en la linea ': 10, 'En la posición ': 153}
{'Tipo de token ': 'NUMENTERO', 'Valor ': 1, 'Esta en la linea ': 10, 'En la posición ': 154}
{'Tipo de token ': 'PARFIN', 'Valor ': ')', 'Esta en la linea ': 10, 'En la posición ': 155}
{'Tipo de token ': 'OP_MAS', 'Valor ': '+', 'Esta en la linea ': 10, 'En la posición ': 157}
{'Tipo de token ': 'VARIABLE', 'Valor ': 'fibonacci', 'Esta en la linea ': 10, 'En la posición ': 159}
{'Tipo de token ': 'PARIN', 'Valor ': '(', 'Esta en la linea ': 10, 'En la posición ': 168}
{'Tipo de token ': 'VARIABLE', 'Valor ': 'x', 'Esta en la linea ': 10, 'En la posición ': 169}
{'Tipo de token ': 'OP_MENOS', 'Valor ': '-', 'Esta en la linea ': 10, 'En la posición ': 170}
{'Tipo de token ': 'NUMENTERO', 'Valor ': 2, 'Esta en la linea ': 10, 'En la posición ': 171}
{'Tipo de token ': 'PARFIN', 'Valor ': ')', 'Esta en la linea ': 10, 'En la posición ': 172}
{'Tipo de token ': 'RETORNAR', 'Valor ': 'retorna', 'Esta en la linea ': 11, 'En la posición ': 182}
{'Tipo de token ': 'VARIABLE', 'Valor ': 'fibo', 'Esta en la linea ': 11, 'En la posición ': 190}
{'Tipo de token ': 'LLAFIN', 'Valor ': '}', 'Esta en la linea ': 12, 'En la posición ': 199}
{'Tipo de token ': 'IMPRIMIR', 'Valor ': 'imprimir', 'Esta en la linea ': 13, 'En la posición ': 201}
{'Tipo de token ': 'PARIN', 'Valor ': '(', 'Esta en la linea ': 13, 'En la posición ': 209}
{'Tipo de token ': 'VARIABLE', 'Valor ': 'fibonacci', 'Esta en la linea ': 13, 'En la posición ': 210}
{'Tipo de token ': 'PARIN', 'Valor ': '(', 'Esta en la linea ': 13, 'En la posición ': 219}
{'Tipo de token ': 'NUMENTERO', 'Valor ': 4, 'Esta en la linea ': 13, 'En la posición ': 220}
{'Tipo de token ': 'PARFIN', 'Valor ': ')', 'Esta en la linea ': 13, 'En la posición ': 221}
{'Tipo de token ': 'PARFIN', 'Valor ': ')', 'Esta en la linea ': 13, 'En la posición ': 222}
{'Tipo de token ': 'LLAFIN', 'Valor ': '}', 'Esta en la linea ': 14, 'En la posición ': 224}
\end{lstlisting}





\section{URL de Repositorio Github}
    \begin{itemize}
    \item URL del Repositorio GitHub:
		\item \url{https://github.com/AJTM06/compiladores.git}
	\end{itemize}

\end{document}